\documentclass[11pt]{amsart}
\usepackage{geometry}                % See geometry.pdf to learn the layout options. There are lots.
\geometry{letterpaper}                   % ... or a4paper or a5paper or ... 
%\geometry{landscape}                % Activate for for rotated page geometry
%\usepackage[parfill]{parskip}    % Activate to begin paragraphs with an empty line rather than an indent
\usepackage{graphicx}
\usepackage{amssymb}
\usepackage{epstopdf}
\DeclareGraphicsRule{.tif}{png}{.png}{`convert #1 `dirname #1`/`basename #1 .tif`.png}

\title{Multivariate Hessian Chain rule}
\author{The Author}
%\date{}                                           % Activate to display a given date or no date

\begin{document}
\maketitle
%\section{}
%\subsection{}
Depending on how the tensors are combined, I think your expression might make sense now.

I've used $x$, $y$ and $z$ for $C$, $B$ and $A$ respectively in the work below, hope it makes sense:

Let $x\in \mathbb{R}^m$ and $y \in \mathbb{R}^n$. Let $y$ depend on $x$ so we can find the Jacobian of $y$ as a function of $x$.

$$
J_x y \in \mathbb{R}^{n \times m} \qquad
\left. J_{x} y \right|_{i,j}
= \frac{\partial y_{i}}{\partial x_{j}}
$$

where $J|_{i,j}$ denotes the $(i,j)$ element of the Jacobian.

Let $z \in \mathbb{R}^{k}$ be a function of $y$. We can also find the Jacobian of $z$ as a function of $y$.

$$
J_y z \in \mathbb{R}^{k \times n} \qquad
\left.
J_{y} z \right|_{t,i}
= \frac{\partial z_{t}}{\partial y_{i}}
$$

We can view $z$ as a function of $x$ by composition

$$
x \rightarrow y \rightarrow z \qquad \qquad
\mathbb{R}^{m} \rightarrow
\mathbb{R}^{n} \rightarrow
\mathbb{R}^{k}
$$

The Jacobian of $z$ as a function of $x$:


$$
J_x z \in \mathbb{R}^{k \times m} \qquad
\left.
J_{x} z \right|_{t,j}
= \frac{\partial z_{t}}{\partial x_{j}}
$$

The term

$$
\frac{\partial z_{t}}{\partial x_{j}}
$$

Can be written as a sum

$$
\frac{\partial z_{t}}{\partial x_{j}}
=
\sum_{i=1}^{n}
\frac{\partial z_{t}}{\partial y_{i}}
\frac{\partial y_{i}}{\partial x_{j}}
$$

Let's drop the summation and use Einstein notation, i.e. we assume summation over indices that are repeated, so we can write

$$
\frac{\partial z_{t}}{\partial x_{j}}
=
\frac{\partial z_{t}}{\partial y_{i}}
\frac{\partial y_{i}}{\partial x_{j}}
$$

This basically represents matrix multiplication.
The Jacobians are matrices (2-tensors) and we have

$$
J_{x} z = J_{y} z \; \; J_{x} y
$$


What about the Hessian? - is there a chain rule for Hessians?


The Hessian of $y$ as a function of $x$ is

$$
H_{x} y = \nabla_{x} J_{x} y
$$

It is a 3-tensor
$$
\left.
H_{x} y \right\vert_{q,i,j} = 
\frac{\partial}{\partial x_{q}}
J_{x} y \vert_{i,j}
$$


$$
\left.
H_{x} y \right\vert_{q,i,j} = 
\frac{\partial}{\partial x_{q}}
\frac{\partial y_{i}}{\partial x_{j}}
=
\frac{\partial^{2} y_{i}}{\partial x_{q} \, x_{j}}
$$


The Hessian of $z$ as a function of $y$ is

$$
\left.
H_{y} z \right\vert_{r,t,i} = 
\frac{\partial}{\partial y_{r}}
\frac{\partial z_{t}}{\partial y_{i}}
=
\frac{\partial^{2} z_{t}}{\partial y_{r} \, y_{i}}
$$

What about the Hessian of $z$ as a function $x$?




$$
\left.
H_{x} z \right\vert_{s,t,j} = 
\frac{\partial}{\partial x_{s}}
\frac{\partial z_{t}}{\partial x_{j}}
=
\frac{\partial^{2} z_{t}}{\partial x_{s} \, x_{j}}
$$

We can use the expression derived earlier for $\partial z_{t}/\partial x_{j}$ which is an element in the Jacobian of $z$ with respect to $x$:

$$
\left.
H_{x} z \right\vert_{s,t,j} = 
\frac{\partial}{\partial x_{s}}
\left[
\frac{\partial z_{t}}{\partial y_{i}}
\frac{\partial y_{i}}{\partial x_{j}}
\right]
$$

Remembering that we are using the convention where repeated indices denote indices over which we should sum. The evaluates with the product rule to the following

$$
\begin{aligned}
\left.
H_{x} z \right\vert_{s,t,j} &
= 
\frac{\partial}{\partial x_{s}}
\left[
\frac{\partial z_{t}}{\partial y_{i}}
\frac{\partial y_{i}}{\partial x_{j}}
\right]
\\
&=
\frac{\partial z_{t}}{\partial y_{i}}
\;
\frac{\partial}{\partial x_{s}}
\frac{\partial y_{i}}{\partial x_{j}}
+
\frac{\partial}{\partial x_{s}}
\frac{\partial z_{t}}{\partial y_{i}}
\;
\frac{\partial y_{i}}{\partial x_{j}}
\\
&=
\frac{\partial z_{t}}{\partial y_{i}}
\,
\frac{\partial^{2} y_{i}}{\partial x_{s} x_{j}}
+
\frac{\partial}{\partial x_{s}}
\frac{\partial z_{t}}{\partial y_{i}}
\,
\frac{\partial y_{i}}{\partial x_{j}}
\end{aligned}
$$

The first part of the second term contains components of all three variables, let's focus on it

$$
\frac{\partial}{\partial x_{s}}
\frac{\partial z_{t}}{\partial y_{i}}
$$

Assuming equality of mixed partials, we can write it as

$$
\frac{\partial}{\partial y_{i} }
\frac{\partial z_{t}}{\partial x_{s}}
$$

Repeating the substitution made earlier,

$$
\begin{aligned}
\frac{\partial}{\partial y_{i} }
\frac{\partial z_{t}}{\partial x_{s}}
&=
\frac{\partial}{\partial y_{i} }
\left[
\frac{\partial z_{t}}{\partial y_{u}}
\frac{\partial y_{u}}{\partial x_{s}}
\right]
\\
&=
\frac{\partial}{\partial y_{i} }
\frac{\partial z_{t}}{\partial y_{u}}
\;
\frac{\partial y_{u}}{\partial x_{s}}
+
\frac{\partial z_{t}}{\partial y_{u}}
\;
\frac{\partial}{\partial y_{i} }
\frac{\partial y_{u}}{\partial x_{s}}
\\
&= 
\frac{\partial^{2} z_{t}}{\partial y_{i} y_{u}}
\;
\frac{\partial y_{u}}{\partial x_{s}}
+
\frac{\partial z_{t}}{\partial y_{u}}
\;
\frac{\partial}{\partial y_{i} }
\frac{\partial y_{u}}{\partial x_{s}}
\\
&= 
\frac{\partial^{2} z_{t}}{\partial y_{i} y_{u}}
\;
\frac{\partial y_{u}}{\partial x_{s}}
+
\frac{\partial z_{t}}{\partial y_{u}}
\;
\frac{\partial}{\partial x_{s} }
\frac{\partial y_{u}}{\partial y_{i} }
\\
&= 
\frac{\partial^{2} z_{t}}{\partial y_{i} y_{u}}
\;
\frac{\partial y_{u}}{\partial x_{s}}
+
\frac{\partial z_{t}}{\partial y_{y}}
\;
\frac{\partial}{\partial x_{s} }
\delta_{i,u}
\\
&=
\frac{\partial^{2} z_{t}}{\partial y_{i} y_{u}}
\;
\frac{\partial y_{u}}{\partial x_{s}}
+
\frac{\partial z_{t}}{\partial y_{u}}
\;
0
\\
&=
\frac{\partial^{2} z_{t}}{\partial y_{i} y_{u}}
\;
\frac{\partial y_{u}}{\partial x_{s}}
\end{aligned}
$$

Now replace this expression into the second term of the earlier equation for the Hessian of $z$ with respect to $x$

$$
\begin{aligned}
\left.
H_{x} z \right\vert_{s,t,j} 
&= 
\frac{\partial z_{t}}{\partial y_{i}}
\,
\frac{\partial^{2} y_{i}}{\partial x_{s} x_{j}}
+
\frac{\partial}{\partial x_{s}}
\frac{\partial z_{t}}{\partial y_{i}}
\,
\frac{\partial y_{i}}{\partial x_{j}}
\\
&= 
\frac{\partial z_{t}}{\partial y_{i}}
\,
\frac{\partial^{2} y_{i}}{\partial x_{s} x_{j}}
+
\frac{\partial^{2} z_{t}}{\partial y_{i} y_{u}}
\;
\frac{\partial y_{u}}{\partial x_{s}}
\,
\frac{\partial y_{i}}{\partial x_{j}}
\end{aligned}
$$

Now we can replace the partial derivative notation with the Jacobian/Hessian notation to get an expression for the Hessian of $z$ as a function of $x$ in terms of the intermediate Jacobians/Hessians:


$$
\begin{aligned}
\left.
H_{x} z \right\vert_{s,t,j} 
&= 
\frac{\partial z_{t}}{\partial y_{i}}
\,
\frac{\partial^{2} y_{i}}{\partial x_{s} x_{j}}
+
\frac{\partial^{2} z_{t}}{\partial y_{i} y_{u}}
\;
\frac{\partial y_{u}}{\partial x_{s}}
\,
\frac{\partial y_{i}}{\partial x_{j}}
\\
&= 
\left.
J_{y} z \right|_{t,i}
\,
\left.
H_{x} y \right|_{s,i,j}
+
\left.
H_{y} z \right|_{i,t,u}
\;
\left.
J_{x} y \right|_{{u,s}}
\,
\left.
J_{x} y \right|_{i,j}
\end{aligned}
$$

So for the first term:

$$
\left.
J_{y} z \right|_{t,i}
\,
\left.
H_{x} y \right|_{s,i,j}
$$
we evaluate the tensor product along the second dimension of the Jacobian of $z$ and the second dimension of the Hessian of $y$. This is because the shared index $i$ appears second for $J_{y} z$ and second for $H_{x} y$. The Numpy tensordot function accepts an argument to set which dimensions to carry out the product on.

For the second term
$$
\left.
H_{y} z \right|_{i,t,u}
\;
\left.
J_{x} y \right|_{{u,s}}
\,
\left.
J_{x} y \right|_{i,j}
$$
There are two tensor products and the shared indices indicate over which dimensions to take the products.



































\end{document}  