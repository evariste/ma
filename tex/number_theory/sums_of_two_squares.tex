\documentclass[11pt]{amsart}
\usepackage{geometry}                % See geometry.pdf to learn the layout options. There are lots.
\geometry{letterpaper}                   % ... or a4paper or a5paper or ... 
%\geometry{landscape}                % Activate for for rotated page geometry
%\usepackage[parfill]{parskip}    % Activate to begin paragraphs with an empty line rather than an indent
\usepackage{graphicx}
\usepackage{amssymb}
\usepackage{amsthm}

\usepackage{epstopdf}
\usepackage{hyperref}


\DeclareGraphicsRule{.tif}{png}{.png}{`convert #1 `dirname #1`/`basename #1 .tif`.png}


\def\0{{\bf 0}}
\def\A{{\bf A}}
\def\A{{\bf A}}
\def\B{{\bf B}}
\def\C{{\bf C}}
\def\D{{\bf D}}
\def\E{{\bf E}}
\def\F{{\bf F}}
\def\G{{\bf G}}
\def\H{{\bf H}}
\def\I{{\bf I}}
\def\J{{\bf J}}
\def\K{{\bf K}}
\def\K{{\bf K}}
\def\L{{\bf L}}
\def\M{{\bf M}}
\def\N{{\bf N}}
\def\P{{\bf P}}
\def\Q{{\bf Q}}
\def\S{{\bf S}}
\def\R{{\bf R}}
\def\T{{\bf T}}
\def\U{{\bf U}}
\def\V{{\bf V}}
\def\W{{\bf W}}
\def\W{{\bf W}}
\def\X{{\bf X}}
\def\Y{{\bf Y}}
\def\Z{{\bf Z}}

\def\a{{\bf a}}
\def\b{{\bf b}}
\def\c{{\bf c}}
\def\d{{\bf d}}
\def\e{{\bf e}}
\def\f{{\bf f}}
\def\g{{\bf g}}
\def\h{{\bf h}}
\def\k{{\bf k}}
\def\n{{\bf n}}
\def\l{{\bf l}}
\def\p{{\bf p}}
\def\q{{\bf q}}
\def\s{{\bf s}}
\def\t{{\bf t}}
\def\u{{\bf u}}
\def\v{{\bf v}}
\def\w{{\bf w}}
\def\x{{\bf x}}
\def\y{{\bf y}}
\def\z{{\bf z}}

\def\cA{{\cal A}}
\def\cB{{\cal B}}
\def\cC{{\cal C}}
\def\cD{{\cal D}}
\def\cE{{\cal E}}
\def\cF{{\cal F}}
\def\cG{{\cal G}}
\def\cH{{\cal H}}
\def\cI{{\cal I}}
\def\cM{{\cal M}}
\def\cN{{\cal N}}
\def\cO{{\cal O}}
\def\cP{{\cal P}}
\def\cQ{{\cal Q}}
\def\cR{{\cal R}}
\def\cS{{\cal S}}
\def\cT{{\cal T}}
\def\cV{{\cal V}}

\def\basis{\theta}
\def\Basis{\Theta}

\def\seg{\pi}

\newenvironment{algo}{\begin{list}{}{\itemsep0mm \parsep0mm \topsep0mm \itemindent0mm \labelwidth0mm \labelsep0mm \partopsep0mm \leftmargin5mm}}{\end{list}}

\newcommand{\vecTwoD}[2]
{
    \ensuremath{
      \begin{pmatrix} #1 \\ #2 \end{pmatrix}
    }
}

\newcommand{\vecThreeD}[3]
{
    \ensuremath{
        \begin{pmatrix} #1 \\ #2 \\ #3 \end{pmatrix}
    }
}

\newcommand{\vecThreeDR}[3]
{
    \ensuremath{
    	\left(
        \begin{array}{r} #1 \\ #2 \\ #3 \end{array}
    \right)
    }
}

\newcommand{\vecFourD}[4]
{
    \ensuremath{
       \begin{pmatrix} #1 \\ #2 \\ #3 \\ #4 \end{pmatrix}
    }
}

\newcommand{\PDeriv}[2]
{
 \ensuremath{
  \frac{\partial {#1}}{\partial {#2}}
 }
}

\newcommand{\PPDeriv}[2]
{
 \ensuremath{
  \frac{\partial^{2} {#1}}{\partial {#2}^{2}}
 }
}

\newcommand{\PPDerivTwoVars}[3]
{
 \ensuremath{
  \frac{\partial^{2} {#1}}{\partial {#2} \partial {#3}}
 }
}

\newcommand{\DDeriv}[2]
{
 \ensuremath{
  \frac{d {#1}}{d {#2}}
 }
}

\newcommand{\DDDeriv}[2]
{
 \ensuremath{
  \frac{d^2 {#1}}{d {#2}^2}
 }
}


\newcommand{\PDerivAt}[3]
{
 \ensuremath{
  \left.
   \frac{\partial {#1}}{\partial {#2}}
  \right \vert_{#3}
 }
}

\newcommand{\PPDerivAt}[3]
{
 \ensuremath{
  \left.
  \frac{\partial^{2} {#1}}{\partial {#2}^{2}}
  \right \vert_{#3}
 }
}



\newcommand{\DDerivAt}[3]
{
 \ensuremath{
  \left.
   \frac{d {#1}}{d {#2}}
  \right \vert_{#3}
 }
}

\DeclareRobustCommand\sfrac[1]{\@ifnextchar/{\@sfrac{#1}}%
                                            {\@sfrac{#1}/}}
\def\@sfrac#1/#2{\leavevmode\kern.1em\raise.5ex
         \hbox{$\m@th\mbox{\fontsize\sf@size\z@
                           \selectfont#1}$}\kern-.1em
         /\kern-.15em\lower.25ex
          \hbox{$\m@th\mbox{\fontsize\sf@size\z@
                            \selectfont#2}$}
}

\newcommand{\Matrix}[1]{\mathcal{#1}}

\newcommand{\Vector}[1]{\mathbf{#1}}

%\newcommand{\uvect}[1]{\textbf{\em #1}}
\newcommand{\uvect}[1]{\boldsymbol{#1}}

\newcommand{\nth}[1]
{
 \ensuremath
 {
  {#1}^{\textrm{th}}
 }
}

\newcommand{\floor}[1]
{
 \ensuremath
 {
  \left\lfloor {#1} \right\rfloor
 }
}

\newcommand{\ceil}[1]
{
 \ensuremath
 {
  \left\lceil {#1} \right\rceil
 }
}


\newcommand{\argmax}{\operatornamewithlimits{argmax}}
\newcommand{\argmin}{\operatornamewithlimits{argmin}}

\newcommand{\Exp}{\operatorname{Exp}}

\newcommand{\grad}{\operatorname{grad}}
\newcommand{\divergence}{\operatorname{div}}
\newcommand{\curl}{\operatorname{curl}}


% \renewcommand{\vec}[1]{\overrightarrow{\mathbf{#1}}}
% \newcommand{\uvec}[1]{\hat{\mathbf{#1}}}
\newcommand{\uvec}[1]{\hat{#1}}

\newcommand{\QED}{\hfill\ensuremath{\square}}%



\newcommand*{\mathcolor}{}
\def\mathcolor#1#{\mathcoloraux{#1}}
\newcommand*{\mathcoloraux}[3]{%
  \protect\leavevmode
  \begingroup
    \color#1{#2}#3%
  \endgroup
}




%%%%%%%%%%%%%%%%%%%%%%%%%%%%%%%%
% Some commands for collaborative editing.

\usepackage{xcolor}
\usepackage[normalem]{ulem}     % for strikethrough
% \sout{text goes here} will strike through text horizontally.
\definecolor{delColor}{RGB}{200,10,10}
\newcommand{\deltext}[1]
{
\color{delColor}
\sout{#1}
\color{black}
}
%
\definecolor{addColor}{RGB}{10,200,10}
\newcommand{\addtext}[1]
{
\color{addColor}
{#1}
\color{black}
}
%%%%%%%%%%%%%%%%%%%%%%%%%%%%%%%%



\title{Product and quotient of numbers that are expressible as the sum of two squares}
\author{The Author}
%\date{}                                           % Activate to display a given date or no date

\newtheorem{theorem}{Theorem}[section]
\newtheorem{corollary}{Corollary}[section]


\usepackage{cleveref}

\begin{document}
\maketitle
%\section{}
%\subsection{}

\section{Product}

\begin{theorem}
Let two integers be expressible as the sum of two integers. I.e. let $m = a^2 + b^2$ and $n = c^2 + d^2$ where $a,b,c,d \in \mathbb{N}$.

Show that the product $mn$ is expressible as the sum of two integers.
\end{theorem}

\begin{proof}

We can write $m = u \overline{u}$ where $u = a + b i$ and $n = v \overline{v}$ where $v = c + d i $.

$mn = u \overline{u} v \overline{v} = uv \overline{uv} = w \overline{w}$ where $w = uv$.

$w = e + f i$ with $e,f \in \mathbb{N}$ so $mn = e^2 + f^2$.

\end{proof}

Specifically, $e + f i = (a + b i) ( c + d i)$ so that $e = ac - bd$ and $f = ad + bc$.

\section{Quotient}

It seems harder to prove a similar result for a quotient, i.e. that if integers $m, q, p$ are such that $m = p q$ and we have $m$ and $p$ both expressible as a sum of two squared integers, then can we prove that $q = m / p$ is also a sum of two squares?

There is a proof of a similar result due to Euler for the result
at
\href{http://eulerarchive.maa.org/docs/translations/E228en.pdf}{this link}.

\begin{theorem}
If $m = pq$ is expressible as the sum of two squares and $p$ is a prime that is the sum of two squares, then $q = m / p$ can be written as the sum of two squares.
\end{theorem}

\begin{proof} \emph{(Euler)}
Let $m = pq = a^2 + b^2$ for $a,b \in \mathbb{N}$.

It is also assumed that The prime $p$ can be written as the sum of two squares, $p = c^2 + d^2$.

We have, for the integer $q$:
$$
q = \frac{a^2 + b^2}{c^2 + d^2}
$$
which means that $c^2 + d^2$ $|$ $a^2 + b^2$.

The following also holds:
$$
\begin{aligned}
c^2 + d^2 \quad &| \quad c^2 ( a^2 + b^2) \\
 c^2 + d^2 \quad &| \quad a^2 ( c^2 + d^2) \\
\end{aligned}
$$
so that $c^2 + d^2$ divides the difference of the right hand terms, i.e. 
$$
c^2 + d^2 \quad
c^2 ( a^2 + b^2) - a^2 ( c^2 + d^2) 
= b^2 c^2 - a^2 d^2
$$

Now $c^2 + d^2 = p$ is a prime so 
$$
p \quad | \quad b^2 c^2 - a^2 d^2 = (bc + ad)  (bc - ad)
$$
means that
$$
\begin{aligned}
p = c^2 + d^2 \quad &| \quad bc + ad  \\
\text{ or } \quad c^2 + d^2 \quad &| \quad bc - ad
\end{aligned}
$$

This means there must be an integer $m$ such that $bc \pm ad = m (c^2 + d^2)$

In the case of $bc + ad = m (c^2 + d^2)$, we can choose integers $x$ and $y$ such that
$$
\begin{aligned}
b &= mc + x \\
a &=  md + y
\end{aligned}
$$

We can substitute these into the expression $bc + ad = m (c^2 + d^2)$ to obtain
$$
(mc + x)c + (md + y)d = m (c^2 + d^2)
$$
which simplifies to
$$
mc^2 + xc + md^2 + yd = m c^2 + m d^2
$$
so that 
$$
xc + yd = 0 \qquad \Rightarrow \qquad
\frac{x}{y} = - \frac{d}{c}
$$

In the other case, where $c^2 + d^2 $ $|$ $bc-ad$ so that  $bc - ad = m (c^2 + d^2)$, we can choose integers $x$ and $y$ such that
$$
\begin{aligned}
b &= mc + x \\
a &=  - md + y
\end{aligned}
$$
which leads to 
$$
(mc + x)c - (- md + y)d = m (c^2 + d^2)
$$
so that 
$$
xc - yd = 0
\qquad \Rightarrow \qquad
\frac{x}{y} = \frac{d}{c}
$$

So both cases lead to either $x/y = -d/c$ or $x/y = d/c$.

Recall that $c^2 + d^2 = p$ which is a prime so $c$ and $d$ cannot have a common factor greater than one.  This means the fraction $d/c$ is in its simplest form and therefore, for some integer $n$, we have
$$
\begin{aligned}
x = nd \quad & y = - n c & \quad \text{ if } \quad
 c^2 + d^2 \quad | \quad bc + ad
\\
x = nd \quad & y =  n c & \quad \text{ if } \quad
c^2 + d^2 \quad | \quad bc - ad
\end{aligned}
$$

We can substitute these into the expressions for $a$ and $b$ to obtain

$$
 a = \mp md \pm nc  \qquad b = mc + nd
$$

Now we can write $pq = a^2 + b^2$ as
$$
\begin{aligned}
pq &= a^2 + b^2 \\
&=
(\mp md \pm nc)^2  +  (mc + nd)^2 
\\
&= 
m^2 d^2 - 2 mncd + n^2 c^2 + m^2 c^2 + 2 mncd + n^2 d^2
\\
&= 
m^2 d^2 + n^2 c^2 + m^2 c^2  + n^2 d^2
\\
&= 
( c^2 + d^2 ) (m^2 + n^2) 
\end{aligned}
$$
from which we can deduce that $q = m^2 + n^2$, so $q$ is the sum of two integers as required.

\end{proof}

\section{Quotient revisited}

What about considering the remainder of numbers after division by 4. Squaring an even number gives a multiple of 4 and squaring an odd number gives an answer of the form $4k+1$. Adding two squared integers therefore gives results that are 0, 1 or 2 mod 4.

I.e. it is impossible to add two squares to obtain a result of the form $4 k + 3$.

\vspace{1em}

Consider a number $m$ that is expressible as the sum of two squares. Let $m = a^2 + b^2$. If $m$ is even, then either $a$ and $b$ are both even or they are both odd.

If $a$ and $b$ are both even, then we can write $m = (2k)^2 + (2l)^2$ 
for some integers $k$,$l$ so that $m = 4 (k^2 + l^2)$. This means we can divide $m$ by 4 to obtain a smaller integer $m'$, that is also the sum of two squares $m/4 = m' = k^2 + l^2$.

\vspace{1em}

In general, if $a$ and $b$ are both odd or both even, assume without loss of generality that $a >= b$. We have $a+b$ and $a-b$ are both even, and we can write $m/2$ as the sum of two integers as follows:
$$
\frac{m}{2} = \left(\frac{a+b}{2}\right)^2 + \left(\frac{a-b}{2}\right)^2
$$


\vspace{1em}

So, if $m = a^2 + b^2$ is even, we can find a smaller integer that is also expressible as the sum of two squares (either $m/4$ or $m/2$).


\vspace{1em}

\begin{theorem}
If an even number is the sum of two squares and has an odd factor, then the odd factor is the sum of two squares.
\end{theorem}

\begin{proof}
We can write a factorisation of $m = a^2 + b^2$ as $m = 2^k n$ where $n$ is an odd integer.
Using the method described above, we can obtain a successively decreasing sequence of integers to `remove' all the factors of 2.
$$
m = m_0 , m_1, m_2, \ldots, m_k = m/2^{k} =  n
$$
where $m_i = m_{i-1}/2$ and $m_{i}$ is expressible as the sum of two squares using the formula above.


In this case we end up with the odd factor $n$ which is the sum of two squares.


\end{proof}

\vspace{1em}

For the next parts, we assume as given the following characteristics of Gaussian primes, namely that
\begin{itemize}
\item
If a prime $p$ is of the form $4k+1$, then $p$ can be factored into Gaussian primes as $p = q \, \overline{q}$  where $q = x + y i$ for some integers $x$ and $y$. Other factorisations can be obtained by replacing $q$ with an associated prime $w q$ where $w$ is a unit, i.e. $w = \pm 1$ or $w = \pm i$.
\item
If a prime $p$ is of the form $4 k + 3$ then it is already a Gaussian prime. It has associated primes $-p$, $ip$ and $-ip$.
\end{itemize}


\vspace{1em}

\begin{theorem}
If $m = a^{2} + b^{2}$ and $p$ is a prime factor of $m$ of the form $4k+1$, then $m/p$ is expressible as the sum of two squares.
\end{theorem}

\begin{proof}
We can write $m = (a+b i) ( a - b i)$. Because $p$ is prime and $p = 1 \mod 4$, it has a factorisation $p = q \overline{q}$ 
where $q$ is a Gaussian prime.

The prime $q$ divides the product $(a+b i) ( a - b i)$ so it must divide one of the factors $a+bi$ or $a-bi$. If $q$ divides $(a+bi)$, say,  write $r = (a+bi) / q$. $r$ is a Gaussian integer and so is $\overline{r}$. In fact $\overline{r} = (a-bi) / \overline{q}$. If we write  $r = c + d i$, then 
$$
c^{2} + d^{2} = r \overline{r} = \frac{(a + b i)}{q} \frac{(a - b i)}{\overline{q}}= \frac{a^{2}+b^{2}}{p} = \frac{m}{p}
$$
which shows that $m/p$ is expressible as the sum of two squares.

A similar argument can be used if $q$ divides $a-bi$.
\end{proof}


\vspace{1em}

\begin{theorem}
If $m = a^{2} + b^{2}$ and $p$ is a prime factor of $m$ of the form $4k+3$, then $p$ divides both of $a$ and $b$.
\end{theorem}

\begin{proof}

Write $m = (a + b i) (a - b i)$. $p$ is of the form $p=4k+3$ so $p$ is a Gaussian prime and therefore cannot be factorised into a product of non-units.

 $p$ divides $m$ so, as a Gaussian prime, $p \, | \, a + b i$
or $p \, | \, a - b i$. In either case, $p \, | \, a $ and $p \, | \, b $.

\end{proof}

\vspace{1em}

\begin{corollary}
\label{cor:pSquaredDivM}
If prime $p$ of the form $4k+3$ and divides $m = a^{2} + b^{2}$ then $p^{2}$ divides both $a^{2}$ and $b^{2}$ which means also that $p^{2} \, | \, m$.
\end{corollary}

\begin{theorem}
If prime $p$ of the form $4k+3$ divides $m = a^{2} + b^{2}$, then $p$ must have an even exponent in the factorisation of $m$.
\end{theorem}

\begin{proof}
From~\cref{cor:pSquaredDivM}, write $a = d p^{q}$ and $b = e p^{r}$ where $d$ and $e$ are not divisible by $p$. Assume without loss of generality that $q \le r$.
So 
$$
m  = d^{2} p^{2q} + e^{2} p^{2r} = 
p^{2q} \left( d^{2} + e^{2} p^{2r-2q} \right)
$$
In the above, the term in the brackets is not divisible by $p$. So $p^{2q}$ is a factor of $m$ with even exponent.
\end{proof}


















\end{document}  