\documentclass[11pt]{amsart}
\usepackage{geometry}                % See geometry.pdf to learn the layout options. There are lots.
\geometry{letterpaper}                   % ... or a4paper or a5paper or ... 
%\geometry{landscape}                % Activate for for rotated page geometry
%\usepackage[parfill]{parskip}    % Activate to begin paragraphs with an empty line rather than an indent
\usepackage{graphicx}
\usepackage{amssymb}
\usepackage{epstopdf}
\DeclareGraphicsRule{.tif}{png}{.png}{`convert #1 `dirname #1`/`basename #1 .tif`.png}

\title{Brief Article}
\author{The Author}
%\date{}                                           % Activate to display a given date or no date

\begin{document}
%\maketitle
%\section{}
%\subsection{}

\emph{When does $m+1 \mid 2m$ ?}


Look at the graphs of $y = 2m$ and $ y = m+1$ and spend some time making tables of $m$, $m+1$ and $2m$ to check.

Now consider cases:

\vspace{1em}

Case: $m > 1$

\vspace{1em}

$$
\begin{aligned}
2m - (m+1) = m-1 &> 0
\\
\Rightarrow
2m &> m+1
\end{aligned}
$$

So

$$
m < m+1 < 2m
$$

So $m+1$ is strictly between two multiples of $m$. I.e. $m+1 \nmid m$.

Also, $m+1 > 2 \Rightarrow m+1 \nmid 2$.

Together these give $m+1 \nmid 2m$ for $m>1$.


\vspace{1em}

Case: $m < -3$:

\vspace{1em}

$$
\begin{aligned}
\frac{m}{2} - (m+1) 
&= \frac{m}{2} - m - 1
\\
&= 
- \frac{m}{2} - 1
\\
&> \frac{3}{2} - 1 = \frac{1}{2} > 0
\end{aligned}
$$
Which gives $\frac{m}{2} > m$.

So we have
$$
\frac{m}{2} > m+1 > m \Rightarrow m+1 \nmid m
$$ 
Because there cannot be a divisor of $m$ strictly between $m$ and $m/2$. And $m+1 \nmid m \Rightarrow m+1 \nmid 2m$.

\vspace{2em}

So we only need to consider the cases $-3 \le m \le 1$.

Make a table:

\begin{center}
\begin{tabular}{c|c|c|c}
$m$ & $m+1$ & $2 m$ & $ m+1 \mid 2m$? \\
\hline
-3 & -2 & -6 & \checkmark \\
-2 & -1 & -4 & \checkmark \\
-1 & 0  & -2 & $\times$ \\
0 &  1  &  0 & \checkmark\\
1 &  2  &  2 & \checkmark\\
\end{tabular}
\end{center}








































\end{document}  