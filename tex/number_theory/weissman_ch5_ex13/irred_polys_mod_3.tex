\documentclass[11pt]{amsart}
\usepackage{geometry}                % See geometry.pdf to learn the layout options. There are lots.
\geometry{letterpaper}                   % ... or a4paper or a5paper or ... 
%\geometry{landscape}                % Activate for for rotated page geometry
%\usepackage[parfill]{parskip}    % Activate to begin paragraphs with an empty line rather than an indent
\usepackage{graphicx}
\usepackage{amssymb}
\usepackage{epstopdf}
\usepackage{color}
\usepackage{hyperref}

\DeclareGraphicsRule{.tif}{png}{.png}{`convert #1 `dirname #1`/`basename #1 .tif`.png}

\title{Brief Article}
\author{The Author}
%\date{}                                           % Activate to display a given date or no date

\begin{document}
%\maketitle
\section{Irreducible polynomials modulo 3}
%\subsection{}

Q: What are the irreducible polynomials of degree up to two modulo 3?


Polynomials of degree 0 (scalars) are `units' (as $\pm 1$ and $\pm i$ are units in the Gaussian integers, or $\pm 1, \pm \omega, \pm \omega^2$ are units in the Eisenstein integers).


Degree 1:

The polynomial $x$ is irreducible - $2 x$ is just a unit multiplied by $x$ so is considered equivalent.

$x + 1$ cannot be obtained from $x$ by multiplication so is a new polynomial. Equivalent multiple of $x + 1$ is $2x + 2$.

$x + 2$ is irreducible, it cannot be obtained from $x$ or from $x+1$ by multiplication. An equivalent polynomial is $2 (x + 2) = 2x + 1$.

Summarise the irreducibility of linear terms so far:
\begin{center}
\begin{tabular}{ll}
polynomial & irreducible $\mod 3$? \\
\hline
x & yes \\
x + 1 & yes \\
x + 2 & yes \\
2 x  & no\\
2 x + 1 & no \\
2 x + 2 & no
\end{tabular}
\end{center}


So, only three linear irreducible polynomials mod 3: $x$, $x+1$ and $x+2$.

What about degree two? We can find \emph{reducible} ones by taking products of the three linear terms.



$$
\begin{aligned}
x \cdot x &= x^2  \\
2 \cdot x \cdot x &= 2 x^2
\end{aligned}
$$

\vspace{0.5em}

$$
\begin{aligned}
x \cdot (x+1) &= x^2 + x  \\
2 \cdot x \cdot (x+1) &= 2 x^2 + 2 x
\end{aligned}
$$

\vspace{0.5em}

$$
\begin{aligned}
x \cdot (x+2) &= x^2 + 2 x  \\
2 \cdot x \cdot (x+2) &= 2 x^2 +  x
\end{aligned}
$$

\vspace{0.5em}


$$
\begin{aligned}
(x+1) \cdot (x+1) &= x^2 + 2x + 1 \\
2 \cdot (x+1) \cdot (x+1) &= 2 x^2 + x + 2
\end{aligned}
$$


\vspace{0.5em}

$$
\begin{aligned}
(x+1) \cdot (x+2) &= x^2 + 0x + 2 \\
2 \cdot (x+1) \cdot (x+2) &= 2 x^2 + 0x + 1
\end{aligned}
$$

\vspace{0.5em}

$$
\begin{aligned}
(x+2) \cdot (x+2) &= x^2 + x + 1 \\
2 \cdot (x+2) \cdot (x+2) &= 2 x^2 + 2x + 2
\end{aligned}
$$

Monic polynomials:

\begin{center}
\begin{tabular}{llcl}
\# & poly & irreducible? & factorisation \\
\hline
1 & $x^2 $                & $\times$   & $x \cdot x$ \\
2 & $x^2 \quad \quad + 1$ & \checkmark & $-$ \\
3 & $x^2 \quad \quad + 2$ & $\times$   & $(x+1)(x+2)$\\
4 & $x^2 + \; x $         & $\times$   & $x(x+1)$\\
5 & $x^2 + \; x + 1$      & $\times$   & $(x+2)^2$ \\
6 & $x^2 + \; x + 2$      & $\times$   & $2(x+1)^2$ \\
7 & $x^2 + 2x  $          & $\times$   & $x(x+2)$ \\
8 & $x^2 + 2x + 1$        & $\times$   & $(x+1)^2$ \\
9 & $x^2 + 2x + 2$        & \checkmark & $-$ \\
\end{tabular}
\end{center}

Non-monic:

\begin{center}
\begin{tabular}{llcl}
\# & poly & irreducible? & factorisation \\
\hline
10 & $2x^2 $                 & $\times$   & $2 x \cdot x$ \\
11 & $2x^2 \quad \quad + 1$  & \checkmark & $-$ \\
12 & $2x^2 \quad \quad + 2$  & \checkmark & $-$ $^*$\\
13 & $2x^2 + \; x $          & $\times$   & $2 x (x+2)$ \\
14 & $2x^2 + \; x + 1$       & \checkmark & $-$ \\
15 & $2x^2 + \; x + 2$       & $\times$   &  $2(x+1)^2$\\
16 & $2x^2 + 2x  $           & $\times$   & $2x(x+1)$ \\
17 & $2x^2 + 2x + 1$         & \checkmark & $-$ \\
18 & $2x^2 + 2x + 2$         & $\times$   & $2(x+2)^2$ \\
\end{tabular}
\end{center}

\vspace{1em}

$^*$ The non-monic $2 x^2 + 2$ equals double the monic irreducible polynomial $x^2 + 1$. It is still irreducible into linear factors.



















\end{document}  