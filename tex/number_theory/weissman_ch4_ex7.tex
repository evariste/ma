\documentclass[11pt]{amsart}
\usepackage{geometry}                % See geometry.pdf to learn the layout options. There are lots.
\geometry{letterpaper}                   % ... or a4paper or a5paper or ... 
%\geometry{landscape}                % Activate for for rotated page geometry
%\usepackage[parfill]{parskip}    % Activate to begin paragraphs with an empty line rather than an indent
\usepackage{graphicx}
\usepackage{amssymb}
\usepackage{epstopdf}
\DeclareGraphicsRule{.tif}{png}{.png}{`convert #1 `dirname #1`/`basename #1 .tif`.png}

\title{Brief Article}
\author{The Author}
%\date{}                                           % Activate to display a given date or no date

\begin{document}
\maketitle
%\section{}
%\subsection{}

\section{a}

$\zeta = e^{2 \pi i / 5} = \cos (2 \pi / 5) + i \sin( 2 \pi / 5)$

\section{b}

\textbf{
Show that $1 + \zeta + \zeta^{2} + \zeta^{3} + + \zeta^{4} = 0$
}

\vspace{1em}

We have 

$$
\begin{aligned}
\zeta &= e^{2 \pi i / 5} = \cos (2 \pi / 5) + i \sin( 2 \pi / 5) \\
\zeta^{2} &= e^{4 \pi i / 5} = \cos (4 \pi / 5) + i \sin( 4 \pi / 5) \\
\zeta^{3} &= e^{6 \pi i / 5} = \cos (6 \pi / 5) + i \sin( 6 \pi / 5) \\
\zeta^{4} &= e^{8 \pi i / 5} = \cos (8 \pi / 5) + i \sin( 8 \pi / 5) \\
\end{aligned}
$$

Let $c = \cos (\pi / 5)$ and $s = \sin (\pi / 5)$.

$$
\begin{aligned}
\zeta &=  (2 c^{2} - 1) + i (2 c s) \\
\zeta^{2} &= (-c) + i (s) \\
\zeta^{3} &= (-c) + i (-s) \\
\zeta^{4} &= (2 c^{2} - 1) + i (- 2 c s) \\
\end{aligned}
$$

Substitute these into the expression $1 + \zeta + \zeta^{2} + \zeta^{3} + + \zeta^{4}$ to get

$$
\begin{aligned}
1 + \zeta + \zeta^{2} + \zeta^{3} + + \zeta^{4} 
&= 
1 + 2 (2 c^{2} - 1) + 2(-c)
\\
&=
4 c^{2}  - 2c - 1
\\
&=
0
\end{aligned}
$$

This equals zero because ... See Section \ref{sec:piOverFiveIdentity}.


\section{c}

\textbf{
Show $z = \zeta + \overline{\zeta}$ is a root of a quadratic equation.
}

\vspace{1em}

$z = \zeta + \overline{\zeta}$ so $z = \zeta +  \zeta^{4}$.

$z^{2} = (\zeta +  \zeta^{4})^{2} = \zeta^{2} + 2 \zeta^{5} + \zeta^{8}$

So

$z^{2} = \zeta^{2} + 2  + \zeta^{3}$

$z + z^{2} = \zeta + \zeta^{4} + \zeta^{2} + 2 + \zeta^{3} = 1$
%
using 
$1 + \zeta + \zeta^{2} + \zeta^{3} + + \zeta^{4} = 0$

So

$$
z^{2} + z - 1 = 0
$$

as required.

\section{d}

\textbf{
Compute the value of $z$ and hence the value of $\cos (72^{\circ})$.
}

\vspace{1em}


$$
z = \frac{-1 \pm \sqrt{5}}{2}
$$

$z = \zeta + \overline{\zeta} = 2 \cos (72^{\circ})$ so we only consider the positive square root.

$$
z = \frac{-1 + \sqrt{5}}{2}
$$

so

$$
\cos(72^{\circ}) = \frac{-1 + \sqrt{5}}{4}
$$

NB 

$$
\cos(72^{\circ}) = \frac{-1 + \sqrt{5}}{4} = \frac{1}{ 2 \varphi}
$$




\section{e}

\textbf{
Demonstrate that $\cos(2 \pi / 7)$ is a root of a nonzero cubic polynomial with integer coefficients.
}

Similar reasoning to the above.

Let $\omega$ be the root of $x^{7} = 1$ with $\omega = \cos(2 \pi / 7) + i  \sin(2 \pi / 7)$.

The seventh roots of 1 are $1, \omega, ..., \omega^{6}$. They are arranged on a heptagon. Their sum is zero because, considering them as vectors, and putting them nose to tail, they form another closed heptagon with edge length 1.

I.e. 

$$
\sum_{k=0}^{6} \omega^{k} = 0
$$

Let $z = \omega + \overline{\omega} = \omega + \omega^{6}$.

$z^{2} = (\omega + \omega^{6})^{2} = \omega^{2} + 2 \omega^{7} + \omega^{12}
= \omega^{2} + 2 + \omega^{5}
$

$$
\begin{aligned}
z^{3} &= (\omega + \omega^{6}) (\omega^{2} + 2 + \omega^{5})
\\
&= \omega^{3} + 2 \omega + \omega^{6} + \omega^{8} + 2 \omega^{6} + \omega^{11}
\\
&= \omega^{3} + 2 \omega + \omega^{6} + \omega + 2 \omega^{6} + \omega^{4}
\\
&= 3 \omega + \omega^{3} + \omega^{4} +
3 \omega^{6}
\end{aligned}
$$

Using the expressions for $z$ and $z^{3}$

$$
z^{3} - 2 z =  \omega + \omega^{3} + \omega^{4} + \omega^{6}
$$

Adding the expression for $z^{2}$


$$
\begin{aligned}
z^{3} - 2 z + z^{2 }
&=
\omega + \omega^{3} + \omega^{4} + \omega^{6} + \omega^{2} + 2 + \omega^{5}
\\
&=
1 + 1 + 
\omega + \omega^{2} + \omega^{3} + \omega^{4} + \omega^{5} + \omega^{6}  
\\
&=
1 + 0 = 1
\end{aligned}
$$

So $z = \omega + \overline{\omega}$ is root of the cubic equation $z^{3} - 2 z + z^{2 } - 1 = 0$.

But $z = 2 \cos (2 \pi / 7)$ so that $\cos (2 \pi / 7)$ is also the root of a cubic.

In particular, if $c = \cos (2 \pi / 7)$, then we have

$$
\begin{aligned}
z^{3} - 2 z + z^{2 } - 1 &= 0
\\
(2 c)^{3} - 2 (2c) + (2c)^{2 } - 1 
&= 0
\\
8 c^{3} + 4c^{2} - 4c  - 1 
&= 0
\end{aligned}
$$

Which is a cubic with integer coefficients as required.



\section{Identity for $\cos (36^{\circ})$}
\label{sec:piOverFiveIdentity}

\textbf{
Show that for $\theta = \pi / 5$, we have $4 \cos \theta^2 - 2 \cos \theta - 1 = 0$
}

\vspace{1em}

Consider $\theta = \pi / 5 =  36^{\circ}$ 

$$
\begin{aligned}
\cos(2 \theta) &= 
\cos(\pi - 3 \theta)
\\ 
&= \cos \pi \cos 3 \theta  \; + \; \sin \pi \sin 3 \theta
\\
&= - \cos 3 \theta
\end{aligned}
$$

The formulas for $\cos 2 \theta$ and $\cos 3 \theta$
$$
\cos 2 \theta = \cos^2 \theta - \sin^2 \theta = 2 \cos^2 \theta - 1
$$

$$
\begin{aligned}
\cos 3 \theta &= Re ( \cos \theta + i \sin \theta)^3
\\
&=
\cos^3 \theta - 3 \cos \theta \sin \theta^2
\\
&=
4 \cos^3 \theta - 3 \cos \theta
\end{aligned}$$

Substitute into the expression earlier

$$
\begin{aligned}
\cos(2 \theta) 
&= - \cos 3 \theta
\\
\cos(2 \theta) +
 \cos 3 \theta &=0
\\
2 \cos^2 \theta - 1
+
4 \cos^3 \theta - 3 \cos \theta
&=0
\end{aligned}
$$

Rearrange

$$
4 \cos^3 \theta
+ 2 \cos^2 \theta
 - 3 \cos \theta 
 - 1
=0
$$

$\cos \theta = -1$ satisfies this (because $\cos 2 \theta = - \cos 3 \theta$ is also satisfied by $\theta = \pi$). So we can take out a factor of $(\cos \theta + 1)$ in the above

$$
\begin{aligned}
4 \cos^3 \theta
+ 2 \cos^2 \theta
 - 3 \cos \theta 
 - 1
&=0
\\
(\cos \theta + 1)
(4 \cos \theta^2 - 2 \cos \theta - 1)
&=0
\end{aligned}
$$

So for $\theta = \pi / 5$, we have $4 \cos \theta^2 - 2 \cos \theta - 1 = 0$.

This gives:
$$
\cos (\pi/5)
=
\frac{2 + \sqrt{4 + 16}}{8} = \frac{1 + \sqrt{5}}{4}
$$


N.B. $\cos (\pi / 5) = \varphi / 2$


\end{document}  