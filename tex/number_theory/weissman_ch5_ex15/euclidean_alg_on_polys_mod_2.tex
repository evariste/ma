\documentclass[11pt]{amsart}
\usepackage{geometry}                % See geometry.pdf to learn the layout options. There are lots.
\geometry{letterpaper}                   % ... or a4paper or a5paper or ... 
%\geometry{landscape}                % Activate for for rotated page geometry
%\usepackage[parfill]{parskip}    % Activate to begin paragraphs with an empty line rather than an indent
\usepackage{graphicx}
\usepackage{amssymb}
\usepackage{epstopdf}
\DeclareGraphicsRule{.tif}{png}{.png}{`convert #1 `dirname #1`/`basename #1 .tif`.png}

\title{Euclidean algorithm on polynomials mod 2}
%\author{The Author}
%\date{}                                           % Activate to display a given date or no date
\usepackage{cleveref}

\begin{document}
\maketitle
%\section{}
%\subsection{}


Q: What is the greatest common divisor of $T^6 + 1$ and $T^{15} + 1$, modulo 2? Use the Euclidean algorithm.

\vspace{2em}

We can carry out long division to obtain

\begin{equation}
\label{eqn:firstStep}
(T^{15} + 1) = (T^9 - T^3) \, (T^6 + 1) + (T^3 + 1)
\end{equation}

This is the first step in the Euclidean Algorithm. 

For the second step, we need to divide $T^6 + 1$ by $T^3 + 1$ and find the remainder.
Doing this gives:
$$
T^6 + 1 = (T^3 - 1) \, (T^3 + 1) - 2
$$

As we are working modulo 2, this can be re-written as
$$
T^6 + 1 = (T^3 - 1) \, (T^3 + 1)
$$
which means that, modulo 2, $T^3 + 1$ divides $T^6 + 1$.
This means, from \cref{eqn:firstStep} that $T^3 + 1$ also divides $T^{15} + 1$.


So, we have 
$$
\text{gcd}(\; T^{15} + 1 \;, \; T^{6} + 1 \;) = T^{3} + 1 \qquad \mod 2
$$


\vspace{2em}

\emph{N.B.} $T=1$ is a zero of $T^{3} + 1$ so it has a factor $(T-1) \equiv (T+1) \mod 2$. $T^{3} + 1$ can be factored into irreducible polynomials mod 2 as
$$
T^{3} + 1 = (T+1) (T^2 + T + 1)
$$











\end{document}  