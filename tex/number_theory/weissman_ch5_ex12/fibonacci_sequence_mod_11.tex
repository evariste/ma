\documentclass[11pt]{amsart}
\usepackage{geometry}                % See geometry.pdf to learn the layout options. There are lots.
\geometry{letterpaper}                   % ... or a4paper or a5paper or ... 
%\geometry{landscape}                % Activate for for rotated page geometry
%\usepackage[parfill]{parskip}    % Activate to begin paragraphs with an empty line rather than an indent
\usepackage{graphicx}
\usepackage{amssymb}
\usepackage{epstopdf}
\DeclareGraphicsRule{.tif}{png}{.png}{`convert #1 `dirname #1`/`basename #1 .tif`.png}

\title{Fibonacci numbers mod 11}
\author{}
%\date{}                                           % Activate to display a given date or no date

\begin{document}
\maketitle
%\section{}
%\subsection{}

Question 12, Chapter 5, Weissman:

Let $F_n$ denote the Fibonacci sequence 0, 1, 1, 2, 3, $\ldots$ Each term is the sum of the previous two. Write $F_0 = 0$, $F_1 = 1$, and $F_2 = 1$, etc. 

Prove that for $n \ge 1$, $F_n \equiv 4^{n-1} (2^{n} - 1) \mod 11$ 

\vspace{2em}

Solution:

\vspace{1em}

Use induction. For $n=1$, $F_{n} = 1$ and $4^{0}(2^{1}-1)=1 \equiv F_{1} \mod 11$. For $n=2$, $F_{n} = 1$ and $4^{1}(2^{2}-1) = 12 \equiv F_{2} \mod 11$.


Assume the relation $F_n \equiv 4^{n-1} (2^{n} - 1) \mod 11$ holds for $n \le k$.

$F_{k+1} = F_{k} + F_{k-1}$ so that $F_{k+1} \equiv F_{k} + F_{k-1} \mod 11$.

Using the induction hypothesis, 
$$
F_{k+1} \equiv 
4^{k-1} (2^{k} - 1) 
+
4^{k-2} (2^{k-1} - 1) 
\mod 11
$$
so that
$$
\begin{aligned}
F_{k+1} 
&\equiv 
4^{k-2} 
\left(
4 (2^{k} - 1) 
+
(2^{k-1} - 1) 
\right)
\mod 11
\\
&\equiv 
4^{k-2} 
\left(
4 \cdot 2^{k} - 4 
+
2^{k-1} - 1
\right)
\mod 11
\\
&\equiv 
4^{k-2} 
\left(
2^{k+2}  
+
2^{k-1} - 5
\right)
\mod 11
\end{aligned}
$$

Inside the bracket above, we can use the relation 
(see separate proof):
$$
2^{k-1} + 2^{k} + 2^{k+2} \equiv 0 \mod 11 
\quad \text{ which gives } \quad
2^{k-1} +  2^{k+2} \equiv - 2^{k} \mod 11 
$$
 

Substituting into the earlier expression gives
$$
\begin{aligned}
F_{k+1} 
&\equiv 
4^{k-2} 
\left(
- 2^{k}  - 5
\right)
\mod 11
\\ &\equiv 
4^{k-2} 
\left(
10 \cdot 2^{k}  - 5
\right)
\mod 11
\\ &\equiv 
5 \cdot 4^{k-2} 
\left(
2^{k+1}  - 1
\right)
\mod 11
\end{aligned}
$$


Now use the relation $5 \cdot 4^{k-2} \equiv 4^{k} \mod 11$ (see separate proof) to obtain
$$
\begin{aligned}
F_{k+1} 
&\equiv 
4^{k} 
\left(
2^{k+1}  - 1
\right)
\mod 11
\end{aligned}
$$
which shows that the formula holds for $n=k+1$ and completes the proof by induction.


\vspace{2em}

Proof of $2^{k-1} + 2^{k} + 2^{k+2} \equiv 0 \mod 11 $:

$$
\begin{aligned}
2^{k-1} + 2^{k} + 2^{k+1} 
& \equiv 
2^{k-1} (1 + 2 + 4) \mod 11
\\
& \equiv 
2^{k-1} \cdot 7 \mod 11
\\
& \equiv 
2^{k-1} \cdot (-4) \equiv - 2^{k+1}\mod 11
\\
\end{aligned}
$$

Which gives 
$$
2^{k-1} + 2^{k} + 2 \cdot 2^{k+1} \equiv 0 \mod 11
\quad \text{ i.e. }
2^{k-1} + 2^{k} + 2^{k+2} \equiv 0 \mod 11
$$



\vspace{2em}

Proof of $5 \cdot 4^{k-2} \equiv 4^{k} \mod 11$:
$$
\begin{aligned}
5 \cdot 4^{k-2} 
& \equiv
- 6 \cdot 4^{k-2} \mod 11
\\
&\equiv
- 3 \cdot 2 \cdot 4^{k-2} \mod 11
\\
&\equiv
8 \cdot 2 \cdot 4^{k-2} \mod 11
\\
&\equiv
16 \cdot 4^{k-2} \mod 11
\\
&\equiv
4^{k} \mod 11
\end{aligned}
$$



















\end{document}  