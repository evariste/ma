\documentclass[11pt]{amsart}
\usepackage{geometry}                % See geometry.pdf to learn the layout options. There are lots.
\geometry{letterpaper}                   % ... or a4paper or a5paper or ... 
%\geometry{landscape}                % Activate for for rotated page geometry
%\usepackage[parfill]{parskip}    % Activate to begin paragraphs with an empty line rather than an indent
\usepackage{graphicx}
\usepackage{amssymb}
\usepackage{epstopdf}
\DeclareGraphicsRule{.tif}{png}{.png}{`convert #1 `dirname #1`/`basename #1 .tif`.png}

\title{Brief Article}
\author{The Author}
%\date{}                                           % Activate to display a given date or no date

\begin{document}
\maketitle
%\section{}
%\subsection{}

\section{XXX}

This question is about deciding whether a sequence defined by a non-linear recursion is cyclic and, if it is, what the period would be.

For example, consider the sequence is defined recursively by 

$$
a_{n+1} =
\frac{1 + a_{n}}{a_{n-1}}
$$

Say we start it off with $a_{1}=1$, $a_{2}= 2$, then the sequence becomes

$$
1, 2, 3, 2, 1, 1, 2, \ldots 
$$

I.e. we get $a_{6} = a_{1}$ and $a_{7} = a_{2}$ so that the sequence is cyclic with period 5.

We can prove that this recursive formula will lead to a cycle by algebraically evaluating expressions for values in the sequence in terms of $a_{1}$ and $a_{2}$.
This brute force approach works but is a bit tedious and seems unlikely to generalise very well. What if we had another similarly defined  sequence with a period of 100 (evaluating terms would be a lot of work), or that was perhaps not even cyclic (evaluating terms would be pointless)?.

So my question is: Is there a more direct method for determining whether such a non-linear recurrence will cycle and, if it does, what the period will be. I guess I mean a method that uses some property(ies) of the recurrence formula directly without the need to evaluate any terms in the sequence.

\newpage

\section{Expressions}


$$
a_{3}
=
\frac{1 + a_{2}}{a_{1}}
$$


$$
a_{4}
=
\frac{1 + a_{3}}{a_{2}} 
=
\frac{1 + a_{1} + a_{2}}{a_{1} a_{2}}
$$



$$
a_{5}
=
\frac{1 + a_{4}}{a_{3}}
=
\frac{1 + a_{1}}
{a_{2}}
$$

$$
a_{6}
=
\frac{1 + a_{5}}{a_{4}} 
=
a_{1}
$$



$$
a_{7}
=
\frac{1 + a_{6}}{a_{5}} 
=
a_{2}
$$


\section{Working out}


$$
\begin{aligned}
a_{3}
=&
\frac{1 + a_{2}}{a_{1}}
\end{aligned}
$$

$$
\begin{aligned}
a_{4}
=&
\frac{1 + a_{3}}{a_{2}} 
\\
=&
\frac{1 + (\frac{1 + a_{2}}{a_{1}})}{a_{2}} 
\\
=&
\frac{1 + a_{1} + a_{2}}{a_{1} a_{2}}
\end{aligned}
$$

$$
\begin{aligned}
a_{5}
=&
\frac{1 + a_{4}}{a_{3}}
\\
=&
\frac{1 + (\frac{1 + a_{1} + a_{2}}{a_{1} a_{2}})}
{(\frac{1 + a_{2}}{a_{1}})}
\\
=&
\frac{1 + a_{1} + a_{2} + a_{1}a_{2}}{a_{1}a_{2}}
\frac{a_{1}}{1 + a_{2}}
\\=&
\frac{1 + a_{1} + a_{2} + a_{1}a_{2}}
{a_{2} (1+a_{2})}
\\=&
\frac{1 + a_{1}}
{a_{2}}
\end{aligned}
$$

$$
\begin{aligned}
a_{6}
=&
\frac{1 + a_{5}}{a_{4}} 
\\
=&
\left(
1 + 
\frac{1 + a_{1}}
{a_{2}}
\right)
\left(
\frac{a_{1} a_{2}}{1 + a_{1} + a_{2}}
\right)
\\
=&
\left(
\frac{1 + a_{1} + a_{2}}{ a_{2} }
\right)
\left(
\frac{a_{1} a_{2}}{1 + a_{1} + a_{2}}
\right)
\\
=&
a_{1}
\end{aligned}
$$



$$
\begin{aligned}
a_{7}
=&
\frac{1 + a_{6}}{a_{5}} 
\\
=&
\frac{1 + a_{1}}{(\frac{1 + a_{1}}
{a_{2}})}
\\
=&
a_{2}
\end{aligned}
$$



\end{document}  





